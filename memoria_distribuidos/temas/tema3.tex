\chapter{Proxy}
\section{Introducción}
El proxy es una interfaz en la cual tiene definidas las cabeceras de las funciones del servidor. El proxy es el encargado de hacer posible la comunicación entre el cliente y el servidor dando transparencia al cliente.

\section{Funcionalidad}
El proxy contiene las siguientes funciones (métodos):
\begin{table}[h!]
	\centering
	\rowcolors{1}{white}{claro}
	\setlength\arrayrulewidth{1.2pt}
	\begin{tabular}{p{7cm} @{\hspace{3mm}}p{8cm}}
		\hline Función & Descripción\\
		\hline descargarFichero(java.lang.String name, int index) & Sube un fichero a la carpeta remota.\\
		getHoraServidor() & El servidor calcula la hora que tiene y la devuelve en milisegundos\\
		listaFicherosCarpetaRemota() & Crea una lista con el nombre de todos los ficheros que existen el la carpeta remota
		del servidor\\
		modificaCarpetaServidor() & Modifica la carpeta remota del servidor\\
		modificaCarpetaServidor(java.lang.String carpetaServidor) & Modifica la carpeta remota del servidor\\
		subirFichero(java.lang.String name, byte[] contenido,boolean actualizar) & Sube un fichero a la carpeta remota\\
		ultimaModificacion(java.lang.String nombre) & Averigua la ultima modificacion de un fichero\\
		\hline
	\end{tabular}
	\caption{Funcionalidad del proxy}
	\label{tabla:funcionalidad_proxy}
\end{table}
\\
Para ver la documentación con más detalle, ver la sección \ref{javadoc}.